\documentclass[10pt]{article}
\pdfpagewidth=\paperwidth
\pdfpageheight=\paperheight
%\addtolength{\hoffset}{-0.5in}
%\addtolength{\voffset}{-0.5in}
%\addtolength{\textwidth}{1in}
%\addtolength{\textheight}{1in}

\title{Parser Generator Generator (PGG): A Tool for \\ Rapid Parser Construction via EBNF Grammars}
\author{Jedidiah McClurg}
\date{October 15, 2012}
\usepackage{amsmath,amscd}
\usepackage{amsthm}
\usepackage{amssymb}
\usepackage{graphicx}
\usepackage{algorithm}
\usepackage{algpseudocode}
\usepackage{enumerate}
\usepackage{multirow}
%\usepackage{tabularx}
\usepackage[top=1.25in, bottom=1.25in, left=1.1in, right=1.1in]{geometry}
\newtheorem{thm}{Theorem}
\newtheorem{lemma}{Lemma}
\newtheorem{defn}{Definition}
\newtheorem{observ}{Observation}
\begin{document}
\maketitle

\begin{abstract}
Designing a Domain Specific Language (DSL) can be quite tedious,
with large amounts of time and effort spent building successive
iterations of the lexer and parser as the language evolves
throughout the development process.
Each time a syntactical change is made, both the lexer/parser and
the Abstract Syntax Tree (AST) data structures and manipulation
functions may change.  This process can become very
time-consuming, and can distract from development of the
language's functionality.  Thus, we have built the Parser
Generator Generator (PGG) system to abstract away this unnecessary
attention to the frontend implementation details.  PGG takes a high-level
grammar specification in an expressive Extended Backus Naur Form (EBNF),
and produces the corresponding OCaml lexer/parser scripts, as
well as AST data structures and manipulation functions.  The
tool can also emit the necessary skeleton code to allow
PGG to be integrated into the build for the project.  This
allows for projects where the entire lexing/parsing/AST
frontend is specified in a highly-readable EBNF form,
and allows DSL developers to quicky get a prototype up and running.
\end{abstract}

\section{Introduction}
This is the introduction

\section{System Overview}
This is the system overview.

\section{EBNF Input Language}
This is about the language

\section{Implementation}
This talks about how the system is implemented.

\section{Case Studies}
\subsection{Simple Modulo Arithmetic Language}
\subsection{PGG Bootstrap}

\section{Related Work}
This is the related work
Simple Grammar Tools
\cite{uiowa_gt} \cite{noonan_grammar}.
\cite{visser2008webdsl} DSL development process!
\cite{heering1989syntax} (SDF = context free syntax + lexical syntax + abstract syntax)
\cite{moonen2001generating} (island grammars)
\cite{klint1993meta} (detailed journal version of the 2001 paper: ASF + SDF, algebraic specification formalism + syntax definition formalism = ways of specifying the language)
\cite{klint2011easy} More recent work by Klint
\cite{van2001sf} (integrated def of lexical, context-free, and abstract syntax.  integration of syntax and semantics. compilation to parser, pretty-printer, etc).
% http://www.cs.wm.edu/~noonan/Grammar/

\section{Conclusion and Future Work}
Blah blah

\section{Aknowledgements}
I would like to thank Aaron Stump and Cesare Tinelli for an educational
semester(s) spent in the Computational Logic Center (CLC) at the University of Iowa.
During this time, I gained valuable experience using OCaml, and had the opportunity
to work with the Center's useful ``Grammar Tool" (GT), which is the inspiration for this project.
Special thanks goes to JJ Meyer, primary author/developer of GT.

\bibliographystyle{plain}
\bibliography{paper}

\end{document}
